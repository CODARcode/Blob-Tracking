\newcommand{\cA}{\mathcal{A}} % algorithm
\newcommand{\R}{\mathbb{R}} % reals
\newcommand{\cD}{\mathcal{D}} % domain
\newcommand{\cC}{\mathcal{C}} % compression
\newcommand{\Xs}{X^*} % local minimizers
\newcommand{\cN}{\mathcal{N}} % neighborhood


\section{Introduction}
\label{sec:intro}

Fusion experiments are performed in a multi-billion-dollar Tokamak reactor to confine plasma in a magnetic field. During each experiment, the Tokamak can enter dangerous disruption states, violently damaging the plasma-facing wall and inducing large magnetic forces in surrounding metallic structures. Significant research has been performed to develop methods to predict such occurrence of disruption, but little research has been reported using graph or topology theory, which can improve prediction speed and accuracy.

During a simulation, a high rate of time-series data (up to a few terabytes per second) are computed at microsecond granularity. The XGC code computes the kinetics at the edge of the Tokamak while the GENE code computes the kinetics in the core. Both codes are 5D kinetic particle codes. The particles are ions and electrons, and the 5 dimensions include 3 position and 2 velocity dimensions. The two codes are coupled in an approximately 3 cm wide area where they exchange fluid properties that have been resampled onto a mesh.

Regions of high turbulence can damage the Tokamak.  These so called ``blobs'' can run along the edge wall down toward the diverter and damage it.  Hence, it is necessary to detect blobs, compute statistics over them to rule out low-energy insignificant ones, and track the high-energy blobs across time steps. The noisiness of the models makes this a challenging computer science problem, for which no good automatic and reliable method exists.

While those raw data can be independently analyzed or combined in a multi-dimensional array, a graph representation of measured data can provide a new way to explain evolution, track changes, detect anomalies, and help uncover hidden cause-and-effect relationships. Tracking blobs with merging and splitting events is needed to understand how events evolve and either become disruptive or remain normal.  How to trace unstable plasma bifurcation or turbulence; how to relate different types of events from different codes; and whether cause-and-effect can be derived from heterogeneous simulation codes are all questions that we want to study.

Hence, we are investigating whether topological visualization algorithms can robustly extract and track blob features in the Tokamak simulations.  We first define blobs as connected components of critical points (maxima and minima) and then associate the blobs over adjacent timesteps.  Such analysis can be achieved by extracting contour trees or Reeb graphs from the input scalar field data.  The topological events, such as births, deaths, and mergers are also visualized with a graph-based representation. %  which are the result of blob tracking done on experimental microscopy rather than simulations.

We are also developing in situ blob tracking method that couples in situ with the XGC/GENE code.  By integrating the analysis code with the ADIOS 2.0 framework, we can get access to the simulation outputs at run time.  We expect to extract finer details in the simulation data with in situ processing than previously possible with posthoc processing.  

The remainder of this report is organized as follows.  Section~\ref{sec:related} summarizes related work in both feature tracking and topology analysis.  Sections~\ref{sec:method} and~\ref{sec:impl} detail our methods and implementations, respectively.  Preliminary results are shown in Section~\ref{sec:results}, and conclusions are drawn in Section~\ref{sec:conclusions}.  