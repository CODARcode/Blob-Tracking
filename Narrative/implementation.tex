\section{Implementation}
\label{sec:impl}

The prototype system is implemented with C++ and depends on ADIOS and VTK.  ADIOS is used for data I/O and in situ data retrieval, and VTK serves as the data structure for various analyses and visualization tasks.  We first load the data from ADIOS, and then convert the input data structures into VTK.  The scalar electrostatic potential field is stored in a unique unstructured mesh structure, and we generate a 3D unstructured mesh to model the torus-shaped mesh for topology analysis.  The data consist of multiple slices that uniformly partition the angular space domain in $[0, 2\pi)$, and each slice shares the same 2D triangular mesh structure.  The triangular mesh consists a list of vertex coordinates and their connections, and the electrostatic potential values are stored in a vector; each component of the vector records the scalar value on a vertex.  In the conversion, we further extrude the 2D mesh to a 3D mesh for topology analysis.  
