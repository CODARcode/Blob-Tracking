\section{Conclusions and Future Works}
\label{sec:conclusions}

In this work, we present a in situ framework to extract topological structures of large scale 5D gyrokinetic Tokamak simulation data.  We define blobs as connected components of critical points, and then extract Reeb graph/contour trees of the data that are preconditioned with the robust principal component analysis.  We then associate the blobs of adjacent timesteps with further analyses.  Topological events, such as merge and splits over time are also identified and visualized for fusion scientists to understanding the dynamics of blobs.  

In the future, we would like to further improve our implementation by replacing the contour trees with Reeb graphs, in order to comprehensively extract blobs in the non-simply connected torus domain.  We must also develop Reeb graph simplification algorithms to simplify the analysis results.  We would also like to provide both automatic and interactive tools for users to further filter important blobs based on geometric measurements.  

\iffalse
Guidelines for selecting persistence threshold.  Learning.  
Tracking over time.
Event visualization.  
In situ integration with ADIOS. 
Further collaboration with fusion scientists.  
\fi