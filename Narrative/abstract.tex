\begin{abstract}

We are developing techniques to robustly extract, track, and visualize blob features in large scale 5D gyrokinetic tokamak simulations, such as XGC.  Blobs, regions of high turbulence that can damage the Tokamak, can run along the edge wall down toward the diverter.  Based on discussions with fusion scientists, we define a blob as the subregion that encircles a local maxima or minima in the scalar derivative of electrostatic potential ($dpot$) field.  Extraction and tracking of blobs can be achieved by contour trees in scalar field topology theory, which can capture comprehensive features and dynamics in the data.  We further simplify the contour trees based on various metrics including persistence, volume, and robustness to only incorporate salient blobs for the analysis.  The dynamics of blobs, including birth, death, and merging events can also be characterized and visualized.  
% topological critical points in every single 2D mesh plane, i.e. maxima, minima, and saddles in scalar field topology theory.  
In our implementation, the inputs are from ADIOS file or in situ staging areas.  We further precondition the data with the robust PCA algorithm, and then extract blobs in the 3D meshes and then associate the blobs over time.
Our data analysis and visualization techniques for blob tracking can help fusion scientists understand the behavior of blob dynamics in greater detail than previously possible. 
% , which are the result of blob tracking done on experimental microscopy rather than simulations.
% Blob extraction and tracking enables the exploration and analysis of high-energy blobs across timesteps.  

\end{abstract}

