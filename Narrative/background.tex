\section{Background}
\label{sec:related}

In this section, we review literature related to topology-based visualization, feature tracking, and in situ visualization.  


\subsection{In situ visualization}

A comprehensive review on in situ visualization is available in~\cite{BauerAACGKMLVWB16}.  In situ visualization is mainly driven by three reasons.  First, there is a disparity between scientific simulations and limited I/O bandwidth, which prevents comprehensive data storage.  Second, domain scientists need to understand simulation outputs in higher temporal resolutions than before.  Our in situ topology-based visualization workflow benefits the analysis of blobs in both aspects.  In the following, we categorize the related work into algorithms, infrastructures, and applications.  

{\bf In situ visualization algorithms} filter online input data and generate posthoc visualization results.  For example, Explorable images~\cite{TikhonovaCM2010}, which can generate new volume rendering results with a small number of rendered images, can be used for in situ visualization.  Similarly, pathtubes can be visualized in situ with explorable images~\cite{YeMM13}.  Ahrens et al. presented an image-based in situ visualization framework for interactive exploration~\cite{AhrensJOPRP14}.  Algorithms are also proposed to choose representative timesteps for in situ visualization~\cite{MalakarVMKHLP15}.  

{\bf In situ visualization infrastructures}\qquad couple the simulations and visualization pipelines.  Production coupling tools include VisIt Libsim~\cite{WhitlockFM11} and ParaView Catalyst~\cite{FabianMTBMGRJ11}.  
Various I/O and staging tools are developed for in situ visualization, such as ADIOS~\cite{LofsteadZKS09}, Damaris/Viz~\cite{DorierSPAS13}, FlexIO~\cite{ZhengZESWDNCAKPY13}, and DataSpaces~\cite{DocanPK12}.  
In addition tool coupling tools, algorithms are proposed to optimize in situ workflows.  For example, GoldRush~\cite{ZhengYHWESAK13} was proposed to use idle in the simulation for in situ processing.  


% Offloading the in situ algorithms into secondary resources, so called in transit visualization, is also a 
% In our study, we decouple the heavyweight vortex analysis in an independent process for parallel and asynchronous processing, in order to reduce the slowdown of the simulation.}

% Instead of using existing frameworks and I/O solutions, the characteristics of both our simulation and analysis algorithms required us to customize our in situ workflow and store the output data in high-performance databases.}

{\bf In situ visualization applications} range from combustion~\cite{YuWGCM10, BennettABGGJKKPPPTYZC12}, climate~\cite{WoodringPSPAH16}, to superconductivities~\cite{GuoPG17}.  All these studies aim to achieve higher temporal resolution of the data in order to extract time-critical features.
In combustion sciences, Bennett et al.~\cite{BennettABGGJKKPPPTYZC12} compute merge trees in combustion simulations by offloading the analyses into secondary resources.  
In ocean-climate models, an in situ work flow was developed to extract and track eddies in MPAS-O simulations~\cite{WoodringPSPAH16}.  
Likewise, Fabian~\cite{Fabian12} detect fragments in explosion simulations in situ.  
Guo et al.~\cite{GuoPG17} proposed an in situ work workflow to track magnetic flux vortices in superconductivity simulations.  


% Woodring et al. developed an in situ workflow to analyze eddies in the study of ocean-climate models.  
% Topologies in combustion data can be extracted by using segmented merge trees in situ~\cite{LandgePGBKCB14}.  
% Yu et al.~\cite{YuWGCM10} visualized volume and particle data in combustion simulations.  



\subsection{Topology-based visualization}
\label{sec:topology}

A comprehensive review on topology-based visualization, which covers scalar, vector, and tensor fields topology is available in~\cite{HeineLHIFSHG16}.  We mainly review techniques related to the level-set based scalar field visualization.  

Formally, the \emph{level set} a real-valued scalar function $f: \Omega\mapsto\mathcal{R}$ is defined as the preimage $f^{-1}(c) = \{x\in\Omega | f(x) = c \}$, which consists one or multiple \emph{contours}.  The topology of the level set may change if we change isovalue $c$, and the topology change can be represented with a \emph{contour tree}~\cite{} if the domain $\Omega$ is simply connected.  In a more general sense, the topology can be represented as Reeb graphs~\cite{Reeb46} if the domain is non-simply connected.  In contour trees or Reeb graphs, each node corresponds to a \emph{critical point}, that is, a local minimum, a local maximum, or a saddle; each node also represents the birth, death, joining, or splitting of one or more contours.  Edges in contour trees or Reeb graphs, so called \emph{arcs}, represent the components between critical points. 

The computation of scalar field topology is well established.  Oesterling et al.~\cite{OesterlingHWMS17} proposed an algorithm to compute contour trees in any dimensions.  Distributed and parallel implementations are also developed to handle large datasets~\cite{MorozovW13}.  More recently, Tierny et al. developed the topology toolkit (TTK)~\cite{TiernyFLGM18}, a generic implementation of many important topology-based analysis algorithms.  

Topology simplification is the key to apply scalar field topology to real-world data analyses, because a large number of critical points could be created by noises.  One of the most efficient topology simplification techniques based on persistence~\cite{EdelsbrunnerLZ02}, which characterizes the ``duration'' of the feature and is defined as the difference of the upper and lower bound of a arc.  Contour trees and Reeb graphs can be simplified by removing least persistent arcs or critical point pairs.  
In addition to persistence, Carr et al.~\cite{CarrSP04} used local geometric measurements, such of areas and volumes, to simplify contour trees.  Pascucci et al.~\cite{Pascucci2004} proposed the branch decomposition, which represents contour trees in a multiresolution manner.  In branch decompositions, contour trees can be simplified by truncating branches with least persistences.  

Scalar field topology has been used for many visualization techniques, especially transfer function designs for volume rendering.  For example, Fujishiro et al.~\cite{FujishiroAT99} proposed an automatic topology-based transfer function design, which uses Reeb graphs to find appropriate global color and opacity mappings for the input data.  Weber et al.~\cite{WeberDCPH07} use different local transfer function for different arcs in the contour trees.  Such local transfer functions can be interactively edited on volume rendered images in a what-you-see-is-what-you-get (WYSWIG) style~\cite{GuoY13}.  An automatic color design for local transfer functions was also studied~\cite{ZhouT09}.  

Scalar field topology is also applied in many scientific domains.  For example, topology-based segmentation is used to extract a wide range of features in combustion simulations~\cite{LandgePGBKCB14}.  Cloud systems can be extracted and visualized with topology information in weather simulations~\cite{DoraiswamyNN13}.  Eddies in oceanic simulations are also tracked in situ, in order to capture comprehensive dynamics of the features~\cite{WoodringPSPAH16}.  Guo et al.~\cite{GuoPG17} proposed a workflow to track magnetic flux vortices in superconductivity simulations.  

In our study, blobs are defined as arcs that correspond to both maxima and minima in the input data.  Reeb trees instead of contour trees are necessary, because the torus-shaped data domain is non-simply connected.  We also use both persistent and geometric measurements to simplify the topology, in order to identify salient blob features.  


\subsection{Feature tracking algorithms}

Feature tracking is a well-studied problem in scientific visualization~\cite{PostVHLD2003}.  In this study, we define feature extraction to mean identifying features in single timesteps and feature tracking to mean associating these features over adjacent timesteps in the input data.  

Feature tracking techniques highly depend on how feature are defined.  In most applications, features are defined as subregions (e.g. subvolumes) of the data; region overlaps are used as priori for tracking.  In flow visualization, Vortex core lines, which have multiple different definitions, can be extracted and tracked with parallel vectors~\cite{PeikertR99} and feature flow fields~\cite{TheiselS03, WeinkaufTGP11}, respectively.  Critical point tracking in vector fields and tensor fields are also studied~\cite{GarthTS04, TricocheSH01, TricocheWSH02, ReininghausKWH12}.  In superconductivity simulations, magnetic flux vortices---topological singularities in the input complex-valued order parameter field---can be tracked by generalizing the mathematical definition to 4D space and time~\cite{GuoPPKG16, GuoPG17, PhillipsGPKG16, PhillipsPKG15}.  

In addition to feature tracking algorithms, the visualization of topological changes of features is also a challenging problem.  The changing dynamics of features can be visualized with story lines~\cite{TanahashiM12} in previous studies~\cite{GuoPPKG16}, and we are going to invent new techniques to visualize the dynamics in fusion simulation data.  


\subsection{Blob tracking in fusion sciences}

In fusion sciences, blob tracking has been studied for both experimental and simulation data on individual 2D poloidal planes.  In experimental datasets, Davis et al.~\cite{DavisKMRSZ14} first locate each local maximum that is greater than a predefined threshold, and then use an ellipse to fit to the contour at the half maximum level; ellipse smaller than a predefined size is then identified as a blob.  The same technique was used in Zweben et al.~\cite{Zweben15} and Churchill et al.~\cite{Churchill17}.  In simulation datasets, Wu et al.~\cite{WuWSCCSCK16} define blobs as local regions that have significant higher temperatures and densities than the surroundings.  A two phase algorithm was proposed to first select candidate points and then extract blobs as the connected components of the candidates.  

In this study, we are going to use scalar field topology theories instead of existing empirical methods to define, extract, and track blobs in the 3D electrostatic turbulence field.  Compared with previous efforts, our method is data-driven and is based on well-established computational topology theories.  The topology-based blob definition is not only compatible with existing connected component based analyses, but also is capable of capturing more comprehensive topological information from the input data than previously possible.  For example, our method makes it possible to track birth/death/split/merge events in situ, which could enable scientific discoveries of how blobs evolve and interact with each other.  


% Robustness~\cite{SkrabaW14}.  
% Tracking vector field critical points~\cite{}.  Vortex core tracking~\cite{TheiselSWHS05}.  Vortex core tracking in scale space~\cite{BauerP02}.  
% Tensor feature tracking~\cite{}.  
% Magnetic flux vortex tracking.
% Critical point tracking~\cite{}.


