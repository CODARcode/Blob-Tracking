\section{Related Work}
\label{sec:related}

Our method is mainly related to feature tracking techniques and scalar field topology theories.  A review on topology-based visualization~\cite{HeineLHIFSHG16}. 



\subsection{Scalar Field Topology}

Reeb graph~\cite{Reeb1946}.  

The topology toolkit~\cite{TiernyFLGM18}.  

Contour tree simplification~\cite{CarrSP04, KreveldOBPS97}, branch decomposition~\cite{Pascucci2004}.

Vector and tensor field simplification~\cite{TricocheSH00, TricocheSH01a, TricocheSHC01}.  

Topology-based transfer function design~\cite{FujishiroAT99}, topology attributes~\cite{TakeshimaTFN04}, symmetry~\cite{ThomasN11}, topology-controlled volume rendering~\cite{WeberDCPH07}, color design~\cite{ZhouT09}, local transfer function~\cite{GuoY13}.  

High-dimensional merge tree~\cite{OesterlingHWMS17}. 




\subsection{Feature Tracking}

Feature tracking is a well-studied problem in scientific visualization~\cite{PostVHLD2003}.  Scalar features~\cite{SilverW98}.  Robustness~\cite{SkrabaW14}.  

Existing blob tracking in fusion plasma~\cite{WuWSCCSCK16}.  Experimental data~\cite{DavisKMRSZ14}.  

Tracking vector field critical points~\cite{GarthTS04}.  Vortex core tracking~\cite{TheiselSWHS05}.  Vortex core tracking in scale space~\cite{BauerP02}.  

Tensor feature tracking~\cite{TricocheSH01, TricocheWSH02}.  

Magnetic flux vortex tracking~\cite{GuoPPKG16, GuoPG17, PhillipsGPKG16, PhillipsPKG15}.

Feature flow fields~\cite{TheiselS03, WeinkaufTGP11}. 

Critical point tracking~\cite{ReininghausKWH12}.

Eddy tracking~\cite{WoodringPSPAH16}.  

Story line or tracking graph visualization~\cite{TanahashiM12, GuoPPKG16}.  



\subsection{In Situ Visualization}

A comprehensive review on in situ visualization is available in~\cite{BauerAACGKMLVWB16}.  

\remark{As summarized by a recent comprehensive review~\cite{BauerAACGKMLVWB16}, in situ visualization is motivated by multiple factors---the disparity between scientific computing and I/O rate, the increased temporal resolution needed for accurate data analysis, and the utilization of all computing resources.  Our in situ workflow benefits the analysis of vortex dynamics in all three aspects.  We categorize the related work on in situ visualization into applications, algorithms, and infrastructures.}

\remark{In situ visualization techniques have been used in various scientific applications.  For example, Yu et al.~\cite{YuWGCM10} visualized volume and particle data in combustion simulations.  Topologies in combustion data can be extracted by using segmented merge trees in situ~\cite{LandgePGBKCB14}.  Woodring et al.~\cite{WoodringPSPAH16} developed an in situ workflow to analyze eddies in the study of ocean-climate models.  Fabian~\cite{Fabian12} used in situ processing to detect fragments in explosion simulations.  All these studies aim to achieve higher temporal resolution of the data in order to extract time-critical features.  In our application, an additional challenge is that the TDGL simulation runs much faster than the vortex analysis algorithms.  We must redesign the workflow and accelerate the algorithms in our in situ processing.}

\remark{Various algorithms are used for efficient in situ visualization.  For example, the performance of in situ volume rendering depends on image compositing algorithms, such as binary swap~\cite{MaPHK1994} and 2-3 swap~\cite{YuWM08}.  Explorable images~\cite{TikhonovaCM2010}, which can generate new volume rendering results with a small number of rendered images, can be used for in situ visualization.  Similarly, pathtubes can be visualized in situ with explorable images~\cite{YeMM13}.  Ahrens et al. presented an image-based in situ visualization framework for interactive exploration~\cite{AhrensJOPRP14}.  In addition to rendering technologies, algorithms are proposed to select optimal numbers of time steps for in situ visualization~\cite{MalakarVMKHLP15}.  GoldRush~\cite{ZhengYHWESAK13} uses idle resources for in situ processing.  
Bennett et al.~\cite{BennettABGGJKKPPPTYZC12} explored a method to offload the computations of merge trees into secondary resources.  In our study, we decouple the heavyweight vortex analysis in an independent process for parallel and asynchronous processing, in order to reduce the slowdown of the simulation.}

\remark{In situ infrastructures are bridges between the simulation code and visualization methods.  Typical examples include ParaView Catalyst~\cite{FabianMTBMGRJ11} and VisIt Libsim~\cite{WhitlockFM11}, which can help scientists and visualization practitioners couple the simulation with production visualization tools.
% Damaris/Viz~\cite{DorierSPAS13}.  
Because visualization algorithms are usually I/O intensive, various I/O solutions are proposed for in situ processing, such as ADIOS~\cite{LofsteadZKS09}, FlexIO~\cite{ZhengZESWDNCAKPY13}, and DataSpaces~\cite{DocanPK12}.  Instead of using existing frameworks and I/O solutions, the characteristics of both our simulation and analysis algorithms required us to customize our in situ workflow and store the output data in high-performance databases.}

% use direct integration for in situ primitive detection and implement the custom in transit workflow.  
% GLEAN~\cite{VishwanathHMP11}.  .  PreDatA~\cite{ZhengADLLKPPSW10}.  .  SCIRun~\cite{ParkerJ95}.  Workflows, decaf, etc.  \remark{TBA: explainations}.

\remark{Recently, a group of visualization researchers started the ``in situ terminology project''~\cite{terminology}, in order to uniformly classify the description of in situ methods.  According to their terminology, the integration type and data access are direct, because the primitive detection code is directly plugged into the TDGL simulations and shares the same address space.  The proximity is twofold: the primitive detector shares the same GPU cores as the simulation, and the vortex analyzer uses CPUs on or off the node.  The synchronization is hybrid: the primitive detector runs synchronously with the simulation, and the vortex analyzer runs asynchronously with the simulation.  The operation controls are not applicable in our workflow, and the output type is explorable.}