\section{Background}
\label{sec:related}

In this section, we review literature related to topology-based visualization, feature tracking, and in situ visualization.  


\subsection{Topology-based Visualization}

A comprehensive review on topology-based visualization, which covers scalar, vector, and tensor fields topology is available in~\cite{HeineLHIFSHG16}.  We mainly review techniques related to the level-set based scalar field visualization.  

Formally, the \emph{level set} a real-valued scalar function $f: \Omega\mapsto\mathcal{R}$ is defined as the preimage $f^{-1}(c) = \{x\in\Omega | f(x) = c \}$, which consists one or multiple \emph{contours}.  The topology of the level set may change if we change isovalue $c$, and the topology change can be represented with a \emph{contour tree}~\cite{} if the domain $\Omega$ is simply connected.  In a more general sense, the topology can be represented as Reeb graphs~\cite{Reeb46} if the domain is non-simply connected.  In contour trees or Reeb graphs, each node corresponds to a \emph{critical point}, that is, a local minimum, a local maximum, or a saddle; each node also represents the birth, death, joining, or splitting of one or more contours.  Edges in contour trees or Reeb graphs, so called \emph{arcs}, represent the components between critical points. 

The computation of scalar field topology is well established.  Oesterling et al.~\cite{OesterlingHWMS17} proposed an algorithm to compute contour trees in any dimensions.  Distributed and parallel implementations are also developed to handle large datasets~\cite{MorozovW13}.  More recently, Tierny et al. developed the topology toolkit (TTK)~\cite{TiernyFLGM18}, a generic implementation of many important topology-based analysis algorithms.  

Topology simplification is the key to apply scalar field topology to real-world data analyses, because a large number of critical points could be created by noises.  One of the most efficient topology simplification techniques based on persistence~\cite{EdelsbrunnerLZ02}, which characterizes the ``duration'' of the feature and is defined as the difference of the upper and lower bound of a arc.  Contour trees and Reeb graphs can be simplified by removing least persistent arcs or critical point pairs.  
In addition to persistence, Carr et al.~\cite{CarrSP04} used local geometric measurements, such of areas and volumes, to simplify contour trees.  Pascucci et al.~\cite{Pascucci2004} proposed the branch decomposition, which represents contour trees in a multiresolution manner.  In branch decompositions, contour trees can be simplified by truncating branches with least persistences.  

Scalar field topology has been used for many visualization techniques, especially transfer function designs for volume rendering.  For example, Fujishiro et al.~\cite{FujishiroAT99} proposed an automatic topology-based transfer function design, which uses Reeb graphs to find appropriate global color and opacity mappings for the input data.  Weber et al.~\cite{WeberDCPH07} use different local transfer function for different arcs in the contour trees.  Such local transfer functions can be interactively edited on volume rendered images in a what-you-see-is-what-you-get (WYSWIG) style~\cite{GuoY13}.  An automatic color design for local transfer functions was also studied~\cite{ZhouT09}.  

Scalar field topology is also applied in many scientific domains.  For example, topology-based segmentation is used to extract a wide range of features in combustion simulations~\cite{LandgePGBKCB14}.  Cloud systems can be extracted and visualized with topology information in weather simulations~\cite{DoraiswamyNN13}.  Eddies in oceanic simulations are also tracked in situ, in order to capture comprehensive dynamics of the features~\cite{WoodringPSPAH16}.  Guo et al.~\cite{GuoPG17} proposed a workflow to track magnetic flux vortices in superconductivity simulations.  

In our study, blobs are defined as arcs that correspond to both maxima and minima in the input data.  Reeb trees instead of contour trees are necessary, because the torus-shaped data domain is non-simply connected.  We also use both persistent and geometric measurements to simplify the topology, in order to identify salient blob features.  


\subsection{Feature Tracking}

Feature tracking is a well-studied problem in scientific visualization~\cite{PostVHLD2003}.  In this study, we define feature extraction to mean identifying features in single timesteps and feature tracking to mean associating these features over adjacent timesteps in the input data.  

Feature tracking techniques highly depend on how feature are defined.  In most applications, features are defined as subregions (e.g. subvolumes) of the data; region overlaps are used as priori for tracking.  In flow visualization, Vortex core lines, which have multiple different definitions, can be extracted and tracked with parallel vectors~\cite{PeikertR99} and feature flow fields~\cite{TheiselS03, WeinkaufTGP11}, respectively.  Critical point tracking in vector fields and tensor fields are also studied~\cite{GarthTS04, TricocheSH01, TricocheWSH02, ReininghausKWH12}.  In superconductivity simulations, magnetic flux vortices---topological singularities in the input complex-valued order parameter field---can be tracked by generalizing the mathematical definition to 4D space and time~\cite{GuoPPKG16, GuoPG17, PhillipsGPKG16, PhillipsPKG15}.  

Blob tracking has been studied for both experimental and simulation data in fusion sciences.  \remark{TBA}.  Existing blob tracking in fusion plasma~\cite{WuWSCCSCK16}.  Experimental data~\cite{DavisKMRSZ14}.  

In addition to feature tracking algorithms, the visualization of topological changes of features is also a challenging problem.  The changing dynamics of features can be visualized with story lines~\cite{TanahashiM12} in previous studies~\cite{GuoPPKG16}, and we are going to invent new techniques to visualize the dynamics in fusion simulation data.  

% Robustness~\cite{SkrabaW14}.  
% Tracking vector field critical points~\cite{}.  Vortex core tracking~\cite{TheiselSWHS05}.  Vortex core tracking in scale space~\cite{BauerP02}.  
% Tensor feature tracking~\cite{}.  
% Magnetic flux vortex tracking.
% Critical point tracking~\cite{}.



\subsection{In Situ Visualization}

A comprehensive review on in situ visualization is available in~\cite{BauerAACGKMLVWB16}.  In situ visualization is mainly driven by three reasons.  First, there is a disparity between scientific simulations and limited I/O bandwidth, which prevents comprehensive data storage.  Second, domain scientists need to understand simulation outputs in higher temporal resolutions than before.  Our in situ topology-based visualization workflow benefits the analysis of blobs in both aspects.  In the following, we categorize the related work into algorithms, infrastructures, and applications.  

\remark{Various algorithms are used for efficient in situ visualization.  Explorable images~\cite{TikhonovaCM2010}, which can generate new volume rendering results with a small number of rendered images, can be used for in situ visualization.  Similarly, pathtubes can be visualized in situ with explorable images~\cite{YeMM13}.  Ahrens et al. presented an image-based in situ visualization framework for interactive exploration~\cite{AhrensJOPRP14}.  In addition to rendering technologies, algorithms are proposed to select optimal numbers of time steps for in situ visualization~\cite{MalakarVMKHLP15}.  GoldRush~\cite{ZhengYHWESAK13} uses idle resources for in situ processing.  
Bennett et al.~\cite{BennettABGGJKKPPPTYZC12} explored a method to offload the computations of merge trees into secondary resources.  In our study, we decouple the heavyweight vortex analysis in an independent process for parallel and asynchronous processing, in order to reduce the slowdown of the simulation.}

\remark{In situ infrastructures are bridges between the simulation code and visualization methods.  Typical examples include ParaView Catalyst~\cite{FabianMTBMGRJ11} and VisIt Libsim~\cite{WhitlockFM11}, which can help scientists and visualization practitioners couple the simulation with production visualization tools.
% Damaris/Viz~\cite{DorierSPAS13}.  
Because visualization algorithms are usually I/O intensive, various I/O solutions are proposed for in situ processing, such as ADIOS~\cite{LofsteadZKS09}, FlexIO~\cite{ZhengZESWDNCAKPY13}, and DataSpaces~\cite{DocanPK12}.  Instead of using existing frameworks and I/O solutions, the characteristics of both our simulation and analysis algorithms required us to customize our in situ workflow and store the output data in high-performance databases.}

Various scientific applications including combustion~\cite{YuWGCM10}, climate~\cite{WoodringPSPAH16} to superconductivities~\cite{GuoPG17}, benefit from in situ visualization techniques.  
\remark{For example, Yu et al.~\cite{YuWGCM10} visualized volume and particle data in combustion simulations.  Topologies in combustion data can be extracted by using segmented merge trees in situ~\cite{LandgePGBKCB14}.  Woodring et al.~\cite{WoodringPSPAH16} developed an in situ workflow to analyze eddies in the study of ocean-climate models.  Fabian~\cite{Fabian12} used in situ processing to detect fragments in explosion simulations.  All these studies aim to achieve higher temporal resolution of the data in order to extract time-critical features.} 
% In our application, an additional challenge is that the TDGL simulation runs much faster than the vortex analysis algorithms.  We must redesign the workflow and accelerate the algorithms in our in situ processing.}


% use direct integration for in situ primitive detection and implement the custom in transit workflow.  
% GLEAN~\cite{VishwanathHMP11}.  .  PreDatA~\cite{ZhengADLLKPPSW10}.  .  SCIRun~\cite{ParkerJ95}.  Workflows, decaf, etc.  \remark{TBA: explainations}.

% \remark{Recently, a group of visualization researchers started the ``in situ terminology project''~\cite{terminology}, in order to uniformly classify the description of in situ methods.  According to their terminology, the integration type and data access are direct, because the primitive detection code is directly plugged into the TDGL simulations and shares the same address space.  The proximity is twofold: the primitive detector shares the same GPU cores as the simulation, and the vortex analyzer uses CPUs on or off the node.  The synchronization is hybrid: the primitive detector runs synchronously with the simulation, and the vortex analyzer runs asynchronously with the simulation.  The operation controls are not applicable in our workflow, and the output type is explorable.}